\chapter{Conclusione}
In conclusione possiamo dire che le blockchain sono molto interessanti per una gestione alternativa sui dati privati all'interno di settori come quelli bancari e finanziari. Anche la sicurezza, i trasporti urbani, così come il comparto della beneficienza, della fabbricazione, del monitoraggio e sanitario potrebbero essere trasformati dal più vasto impiego dei registri distribuiti su cui si basa l'architettura sperimentata.
Gli analisti, negli ultimi anni, sono arrivati a ipotizzare che gran parte delle industrie potrebbe trarre benefici più o meno eclatanti dall’impiego dei distributed ledger. Ci sono parecchi casi di utilizzo commerciale delle blockchain, con transazioni che vengono automaticamente verificate e organizzate da una piattaforma decentralizzata che non richiede la supervisione di un ente o di un soggetto centrale (super partes), pur garantendo la resistenza a manomissioni e frodi. Usi di tale genere possono essere facilitati tramite l'impiego di progetti come quelli basati su Hyperledger Fabric che, in molti contesti, facilitano la gestione della struttura dell'architettura e della logica di business sia sotto un punto di vista programmatico che applicativo. Infine, l'utilizzo di Hyperledger Fabric riesce a gestire programmaticamente e in totale semplicità tutte le operazioni di base per il rispetto dei criteri apportati all'interno della GDPR, tale caratteristica ha fatto si che molte aziende si interessassero all'adattamento della propria infrastruttura verso tecnologie innovative come quelle offerte dai progetti del dominio di Hyperledger.
\newpage